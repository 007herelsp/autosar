\documentclass[twocolumn]{article}
\usepackage{verbatim}
\usepackage[utf8x]{inputenc}
\title{A semantics of core AUTOSAR}
\author{Johan Nordlander \and Patrik Jansson}

\begin{document}
\maketitle
\begin{abstract}
  
\end{abstract}

\section{Introduction}
\label{sec:Intro}

Enkla exempel redan tidigt (informellt)

Kort om begränsningar och contributions

\section{A calculus for AUTOSAR software components}
\label{sec:Calc}

TODO: referenser från AUTOSAR-syntax och koncept i artikeln till AUTSAR-dokumentation (standarden)

TODO: mer detaljer om begränsningar: vad vi inte täcker in i kalkylen

\section{Semantics}
\label{sec:Sem}

TODO: referenser från AUTOSAR-regler i artikeln till AUTSAR-dokumentation (standarden)

\section{Examples}
\label{sec:Examples}

\begin{itemize}
\item Mer detaljer om de enkla exemplen + ett större exempel.
\item några exempel, valda för att illustrera värdet av kalkylen och vår semantik
\end{itemize}

\section{Discussion / results}
\label{sec:Disc}

\begin{itemize}
\item osäkerheter: där det finns alternativa tolkningar
\item förslag på förtydliganden / förbättringar / exempel på oklarheter
\end{itemize}

\section{Conclusions and Future Work}
\label{sec:Conc}

Return to the limitation - what is the next step (which can be lifted more easily) 

\onecolumn
\appendix
\section{Prolog code}
\label{sec:Prolog}

\verbatiminput{../prolog/semantics.pl}

\end{document}

