\documentclass[twocolumn]{article}
\usepackage{verbatim}
\usepackage[utf8x]{inputenc}
\usepackage[a4paper,margin=1.4cm]{geometry}
% \usepackage{minted}
\title{A semantics of core AUTOSAR}
\author{Johan Nordlander \and Patrik Jansson}

\begin{document}
\maketitle
\begin{abstract}
  
\end{abstract}

\section{Introduction}
\label{sec:Intro}

The AUTOSAR standard is an open software component architecture for the automotive industry. Its main purpose is to enable interoperability of software modules among different vendors and on heterogeneous platforms, via an extensive set of standardized interfaces and libraries, and a common software development methodology. 

The standard has a rather wide scope and covers many features normally associated with complex operating systems, like I/O abstraction, concurrency, communication, distribution and real-time predictability. Unlike existing operating systems, however, AUTOSAR is not de facto defined in terms of a particular implementation. Instead, an explicit goal of AUTOSAR is to constitute an abstract specification that allows multiple competing realizations, and even systems built as an assembly of fragments from many different (and competing) vendors. Such a goal naturally puts the focus on the standard specification itself. 

Unfortunately, the AUTOSAR specification is not very rigorous, despite a sheer size of more than one hundred pdf documents and over 12,500 pages of text and UML diagrams in total. It is also not very abstract, in that it makes frequent references to assumed implementation techniques for the purpose of defining its semantics. In practice, the AUTOSAR standard becomes blurred with the specific behavior of one's chosen platform and development tools. And because the standard is open to interpretation, the interoperability of software components across tools and platforms is often seriously hampered. What is more, a single software component cannot easily be studied and understood in isolation, since its interactive behavior is only indirectly defined in terms of the concrete C-code and OS tasks that realize it and every interacting component.

This paper takes an important step towards a remedy to these problems, by contributing a formal specification of a substantial core of the AUTOSAR standard. The formalization covers most of the Software Component Template (SWC) and its accompanying Run-Time Environment (RTE) (Section~\ref{sec:Calc}), and is able to directly express every legal way a system of SWCs can evolve in at run-time on an arbitrarily fast platform (Section~\ref{sec:Sem}). It can thus serve as a basis for both a concrete AUTOSAR implementation on a specific platform (whose supported behaviors must be a subset of those defined by the formal semantics), and a platform-independent AUTOSAR simulator (where behaviors from the legal set can be picked at random). Examples of semantic derivations and their applications are found in Section~\ref{sec:Examples}.

The semantics has been formalized with the intent to accurately capture the informal meaning of the AUTOSAR standard, although mistakes and misunderstandings are certainly both possible and plausible. A formal notation is nevertheless a good starting point for any discussion on the resolution of such issues. The semantics is furthermore written to be unambiguous, except where concurrent execution should allow for more than one observed behavior. At some points, the AUTOSAR documents have been found unclear; here a specific choice has been made but alternative interpretations will be discussed separately (Section~\ref{sec:DiscAmb}). At other points, the standard documents are clear but the resulting semantics is still dubious. These cases will also be emphasized and discussed, together with suggested ways forward (Section~\ref{sec:DiscImp}).


TODO: Limitations

TODO: Simple examples early (informally)


\section{Syntax and prelliminaries}
\label{sec:Calc}



TODO: links from syntax and concepts introduced here to the AUTSAR spec

TODO: details on limitations: what's not covered by the calculus

\section{Semantics}
\label{sec:Sem}

TODO: links from rules and choices made here to the AUTSAR spec

\section{Examples}
\label{sec:Examples}

\begin{itemize}
\item More details on the simple examples + a bigger example.
\item Some examples chosen to illustrate the value of a formal calculus
\end{itemize}

\section{Discussion / results}
\label{sec:Disc}

\subsection{Ambiguities and alternative interpretations}
\label{sec:DiscAmb}

\subsection{Clarification proposals and improvements}
\label{sec:DiscImp}

\section{Conclusions and Future Work}
\label{sec:Conc}

Return to the limitation - what is the next step (which can be lifted more easily) 

\onecolumn
\appendix
\section{Prolog code}
\label{sec:Prolog}

% \inputminted{prolog}{../prolog/semantics.pl}

\verbatiminput{../prolog/semantics.pl}

\end{document}

