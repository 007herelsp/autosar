(Pre-writing)

The semantics uses a few operators to make the Prolog source code more
like a labelled transition system:

The operator \verb+:+ is used to separate different part of a the name of
an entity a bit like ``.'' is used in Haskell and Java.

Different software components have input and output ports and (the
names of) these are wired together with the relation \verb+==>+

Finally the ternary operator ``Before \verb+---+ Label \verb+--->+
After'' is used for the labeled transitions (reductions).

The top level labels are of the form \verb+say(A,L)+,
\verb+hear(A,L)+, and \verb+delta(T)+.

The \verb+L+ can be of the following forms

\begin{verbatim}
   I.R!new()
   I.R!term()              I.R?term()
   I.P.E!inv()             I.P.E?inv()
   I.P.E!rcv(V)            I.P.E?rcv(V)
                           I.P.E?rcv(no_data)
   I.P.E!rd(V)             I.P.E?rd(V)
   I.P.E!snd(V,Res)        I.P.E?snd(V,limit)
                           I.P.E?snd(V,ok)
   I.P.E!up(U)             I.P.E?up(U)
   I.P.E!wr(V)             I.P.E?wr(V)
   I.P.O!call(V,I.P.O,Res) I.P.O?call(V,C,limit)
   I.P.O!call(V,I.P.O,ok)  I.P.O?call(V,C,ok)
   I.P.O!res(V)            I.P.O?res(V)
                           I.P.O?res(no_data)
   I.P.O!ret(V)            I.P.O?ret(V)
   I.S!irvr(V)             I.S?irvr(V)
   I.S!irvw(V)             I.S?irvw(V)
   I.X!enter()             I.X?enter()
   I.X!exit()              I.X?exit()
\end{verbatim}

If we call head of the part after ``!'' or ``?'' the payload, we have
the following payloads (sorted by the part before the punctuation):

\begin{itemize}
\item I.R:   new, term
\item I.P.E: inv, rcv, rd, snd, up, wr
\item I.P.O: call, res, ret
\item I.S:   irvr, irvw
\item I.X:   enter, exit
\end{itemize}

\subsection{Combining reductions}

TODO: prettify and explain the function ``eval''.
\begin{verbatim}
[]          --- hear(_A,_L) --->  []   .
[ P | Ps ]  --- say(A,L)    --->  concat(eval([P2 | Ps2]))  :-   P  --- say(A,L) --->   P2  ,  Ps  --- hear(A,L) --->  Ps2.
[ P | Ps ]  --- say(A,L)    --->  [ P2 | Ps2 ]  :-   P  --- hear(A,L) --->  P2  ,  Ps  --- say(A,L) --->   Ps2 .
[ P | Ps ]  --- hear(A,L)   --->  [ P2 | Ps2 ]  :-   P  --- hear(A,L) --->  P2  ,  Ps  --- hear(A,L) --->  Ps2 .
[ P | Ps ]  --- say(A,L)    --->  [ P  | Ps2 ]  :-   Ps  --- say(A,L) --->   Ps2  ,  ignore((A,L),P)    .
[ P | Ps ]  --- hear(A,L)   --->  [ P  | Ps2 ]  :-   Ps  --- hear(A,L) --->  Ps2  ,  ignore((A,L),P)    .
\end{verbatim}
Think of \verb+ignore((A,L),P) :- P --- hear2(A,L) --->P.+ but implemented to avoid overlap.
\begin{verbatim}
[]          --- delta(_T)   --->  []    .
[ P | Ps ]  --- delta(T)    --->  [ P2 | Ps2 ]  :-   P  --- delta(T) --->   P2  ,    Ps  --- delta(T) --->  Ps2  .
\end{verbatim}

\subsection{Exclusive areas}

Strict stack-based locking enforced: reduction gets stuck otherwise
(TODO: comment about flagging error on attempted "bad" \verb+rte_Exit+?)
\begin{verbatim}
rinst(R:I, C, Xs, rte_Enter(X,K))        --- say(X:I,enter) --->     rinst(R:I, C, [X|Xs], ap(K,ok))  .
rinst(R:I, C, [X|Xs], rte_Exit(X,K))     --- say(X:I,exit) --->      rinst(R:I, C, Xs, ap(K,ok))      .
excl(X:I, free)                          --- hear(X:I,enter) --->    excl(X:I, taken)                    .
excl(X:I, taken)                         --- hear(X:I,exit) --->     excl(X:I, free)                     .
\end{verbatim}

\subsection{Inter-runnable variables}

\begin{verbatim}
rinst(R:I, C, Xs, rte_IrvRead(S,K))      --- say(S:I,irvr(V)) --->     rinst(R:I, C, Xs, ap(K,V))  .
rinst(R:I, C, Xs, rte_IrvWrite(S,K))     --- say(S:I,irvw(_V)) --->    rinst(R:I, C, Xs, ap(K,ok)) .
irv(S:I, V)                              --- hear(S:I,irvr(V)) --->    irv(S:I, V)                    .
irv(S:I, _)                              --- hear(S:I,irvw(V)) --->    irv(S:I, V)                    .
\end{verbatim}

\subsection{Sending/receiving}

\begin{verbatim}
rinst(R:I, C, Xs, rte_Receive(E:P,K))    --- say(E:P:I,rcv(V)) --->        rinst(R:I, C, Xs, ap(K,V)) .
rinst(R:I, C, Xs, rte_Send(E:P,V,K))     --- say(E:P:I,snd(V,Res)) --->    rinst(R:I, C, Xs, ap(K,Res)).

qelem(E:P:I, N, [V|Vs])                  --- hear(E:P:I,rcv(V)) --->       qelem(E:P:I, N, Vs)    .
qelem(E:P:I, N, [])                      --- hear(E:P:I,rcv(no_data)) ---> qelem(E:P:I, N, [])    .
qelem(E:P:I, N, Vs)                      --- hear(A,snd(V,ok)) --->        qelem(E:P:I, N, Vs++[V])
    :-    A==>E:P:I, length(Vs,X), X < N.
qelem(E:P:I, N, Vs)                      --- hear(A,snd(_V,limit)) --->    qelem(E:P:I, N, Vs)
    :-    A==>E:P:I, length(Vs,N)    .
qelem(E:P:I, N, Vs)                      --- hear(A,snd(V,Res)) --->       qelem(E:P:I, N, VS++[V])
    :-    A==>E:P:I, length(Vs,X), X < N, Res \= ok    .
\end{verbatim}

\verb+dataReceived(E:P)+ is a static property of this runnable
  from pending it will move on to spawn an rinst which will execute some "data handler" code.
\begin{verbatim}
runnable(R:I, K, T, _, N)                --- hear(A,snd(_V,ok)) --->   runnable(R:I, K, T, pending, N)
    :-    A==>E:P:I, events(R:I, dataReceived(E:P))    .
runnable(R:I, K, T, Act, N)              --- hear(A,snd(_V,limit)) --->runnable(R:I, K, T, Act, N)
    :-    A==>E:P:I, events(R:I, dataReceived(E:P))    .
\end{verbatim}

\subsection{Reading/writing (unbuffered versions of rcv and snd)}

\begin{verbatim}
rinst(R:I, C, XS, rte_Read(E:P,K))       --- say(E:P:I,rd(V)) --->     rinst(R:I, C, XS, ap(K,V)) .
rinst(R:I, C, XS, rte_Write(E:P,V,K))    --- say(E:P:I,wr(V)) --->     rinst(R:I, C, XS, ap(K,ok)).
delem(E:P:I, _U, V)                      --- hear(E:P:I,rd(V)) --->    delem(E:P:I, false, V)        .
delem(E:P:I, _U, _)                      --- hear(A,wr(V)) --->        delem(E:P:I, true, V)
    :-    A==>E:P:I .
runnable(R:I, K, T, _, N)                --- hear(A,wr(_V)) --->       runnable(R:I, K, T, pending, N)
    :-    A==>E:P:I, events(R:I, dataReceived(E:P))    .
rinst(R:I, C, XS, rte_IsUpdated(E:P,K))  --- say(E:P:I,up(U)) --->     rinst(R:I, C, XS, ap(K,U)) .
rinst(R:I, C, XS, rte_Invalidate(E:P,K)) --- say(E:P:I,inv) --->       rinst(R:I, C, XS, ap(K,ok)).
delem(E:P:I, U, V)                       --- hear(E:P:I,up(U)) --->    delem(E:P:I, U, V)            .
delem(E:P:I, _U, _)                      --- hear(A,inv) --->          delem(E:P:I, true, invalid)
    :-    A==>E:P:I .
\end{verbatim}

\subsection{Calling a server}

\begin{verbatim}
rinst(R:I, C, XS, rte_Call(O:P,V,K))     --- say(O:P:I,call(V,O:P:I,Res)) --->  rinst(R:I, C, XS, ap(K,Res))
    :- serverCallPoint(R:I, async(O:P)) ; Res \= ok   .
rinst(R:I, C, XS, rte_Call(O:P,V,K))     --- say(O:P:I,call(V,O:P:I,ok)) --->   rinst(R:I, C, XS, rte_Result(O:P,K))
    :- serverCallPoint(R:I, sync(O:P))  .
runnable(R:I, K, T, serving(Cs,Vs), N)   --- hear(A,call(V,C,ok)) --->          runnable(R:I, K, T, serving(CS++[C],Vs++[V]), N)
    :-
    A==>O:P:I, events(R:I, operationInvoked(O:P)),
    \+ member(C, Cs)
    .
runnable(R:I, K, T, serving(Cs,Vs), N)   --- hear(A,call(_V,C,limit)) --->      runnable(R:I, K, T, serving(Cs,Vs), N)
    :-
    A==>O:P:I, events(R:I, operationInvoked(O:P)),
    member(C, Cs)
    .
\end{verbatim}

\subsection{Obtaining a server result}

\begin{verbatim}
rinst(R:I, C, XS, rte_Result(O:P,K))  --- say(O:P:I,res(V)) --->        rinst(R:I, C, XS, ap(K,V))  .
rinst(A, O:P:I, [], return(V))        --- say(O:P:I,ret(V)) --->        rinst(A, nil, [], return(void)).
opres(O:P:I, [V|Vs])                  --- hear(O:P:I,res(V)) --->       opres(O:P:I, Vs)               .
opres(O:P:I, [])                      --- hear(O:P:I,res(no_data)) ---> opres(O:P:I, [])
    :-    async_result(O:P:I)   .
opres(O:P:I, Vs)                      --- hear(O:P:I,ret(V)) --->       opres(O:P:I, VS++[V]) .
\end{verbatim}

\subsection{Spawning and terminating}

\begin{verbatim}
rinst(A, nil, [], return(_V))            --- say(A,term) --->     []    .
runnable(A, K, T, Act, N)                --- hear(A,term) --->    runnable(A, K, T, Act, N-1).
\end{verbatim}
Note the "time left" state has to be zero for these rules to fire.

\begin{verbatim}
runnable(A, K, 0, pending, N)            --- say(A,new) --->    [ runnable(A, K, T, idle, N+1)
                                                                , rinst(A, nil, [], ap(K,void)) ]
    :-
    (N == 0 ; canBeInvokedConcurrently(A)),
    minimumStartInterval(A, T)
    .
runnable(A, K, 0, serving([C|Cs],[V|Vs]), N) --- say(A,new) --->   [ runnable(A, K, T, serving(Cs,Vs), N+1)
                                                                   , rinst(A, C, [], ap(K,V)) ]
    :-
    (N == 0 ; canBeInvokedConcurrently(A)),
    minimumStartInterval(A, T)
    .
\end{verbatim}
\subsection{Passing time}

\begin{verbatim}
timer(A, 0)                 --- say(A,tick) --->  timer(A, T)                :-    events(A, timing(T))    .
runnable(A, K, T, _, N)     --- hear(A,tick) ---> runnable(A, K, T, pending, N) .
\end{verbatim}

The "time left" state V is always decreasing in delta(T) steps, and never negative.
\begin{verbatim}
timer(A, V)                 --- delta(T) --->     timer(A, V-T)               :-    V >= T.
runnable(A, K, V, Act, N)   --- delta(T) --->     runnable(A, K, V-T, Act, N) :-    V >= T.
runnable(A, K, 0, Act, N)   --- delta(_T) --->    runnable(A, K, 0, Act, N)   .
rinst(A, C, XS, Code)       --- delta(_T) --->    rinst(A, C, XS, Code)       .
excl(A, V)                  --- delta(_T) --->    excl(A, V)                  .
irv(A, V)                   --- delta(_T) --->    irv(A, V)                  .
qelem(A, N, Vs)             --- delta(_T) --->    qelem(A, N, Vs)            .
delem(A, U, V)              --- delta(_T) --->    delem(A, U, V)             .
opres(A, Vs)                --- delta(_T) --->    opres(A, Vs)               .
\end{verbatim}


\subsection{Ignoring broadcasts}

\begin{verbatim}
ignore(_AL, rinst(_B,_C,_Xs,_K))   .
ignore(_AL, timer(_B,_T))             .

ignore((A,_L), runnable(B, _K, _T, _Act, _N))   :- A \== B.
ignore((A,_L), excl(B,_V))                      :- A \== B.
ignore((A,_L), irv(B,_V))                       :- A \== B.
ignore((A,_L), qelem(B,_N,_Vs))                 :- A \== B, \+ A==>B.
ignore((A,_L), delem(B,_U,_V))                  :- A \== B, \+ A==>B.
ignore((A,_L), opres(B,_Vs))                    :- A \== B, \+ A==>B.
\end{verbatim}


\subsection{Helper predicate: flattening and evaluating reduction results}

\begin{verbatim}
flateval([],              Q, Q)  :- !  .
flateval([ M | P ],       Q, R)  :- !, flateval(P, Q, R1),  flateval(M, R1, R)  .
flateval(rinst(A,C,Xs,M), Q, [ rinst(A,C,Xs,N) | Q ])  :- !, eval(M, N)   .
flateval(M,               Q, [ M               | Q ]).
\end{verbatim}

Experiment with "treelike" process soup
\begin{verbatim}
treeeval([],              [])         :- !  .
treeeval([ M | P ],       [M2 | P2])  :- !, treeeval(M, M2), treeeval(P, P2).
treeeval(rinst(A,C,Xs,M), [ rinst(A,C,Xs,N) | _Q ])  :- !, eval(M, N)   .
treeeval(M,               [ M               | _Q ]).
\end{verbatim}

\subsection{Helper predicate: evaluating terms}

\begin{verbatim}
eval(V, V)              :-    var(V), !    .
eval(ap(T,V), ap(T,V))  :-    var(T), !    .
eval(ap(F,E), R)        :-    eval(F, fn(X,T)), !,    eval(E, V),    X = V,    eval(T, R)    .
eval(ap(T,_V), R)       :- !, eval(T, R)   .
eval(fn(X,T), fn(X,T))  :- ! .
eval(if(E,A,_B), R)     :-    E,    !, eval(A, R)    .
eval(if(E,_A,B), R)     :-    \+ E, !, eval(B, R)    .
eval((A,B), (A1,B1))    :- !, eval(A, A1),     eval(B, B1)    .
eval([A|B], [A1|B1])    :- !, eval(A, A1),     eval(B, B1)    .
eval((A:B), (A1:B1))    :- !, eval(A, A1),     eval(B, B1)    .
eval(E, R)              :-    E =.. [H|As], funWithArgsToEvaluate(H), !, eval(As, Bs), R =.. [H|Bs].
eval(V, V)              :-    atom(V), !  .
eval(E, R)              :-    R is E      .

funWithArgsToEvaluate(H):- name(H,N), (append("rte_",_,N) ; N = "return").
\end{verbatim}
